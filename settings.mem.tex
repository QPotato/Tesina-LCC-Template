\documentclass[a4paper, 11pt, oldcommands, onesided]{memoir}

\usepackage{helvet}
\renewcommand*{\familydefault}{\rmdefault}

% Enumera las subsections
\setsecnumdepth{subsection}
\setcounter{tocdepth}{2} % pone las subsections en el índice

\usepackage{tocloft} %% Editar el ToC
% \setlength{\cftbeforesectionskip}{0.5em}
% % \setlength{\cftbeforesubsectionskip}{1in}
% \setlength{\cftbeforechapterskip}{1em}
% %\setlength{\cftafterskip}{1em}
% %\setlength{\cftparskip}{0em}
% \setlength{\parindent}{1.5em}
% % \setlength{\parskip}{0.3em}
\cftpagenumberson{subsection}

\usepackage[utf8]{inputenc}
%\usepackage[spanish]{babel}
\usepackage[spanish]{babel}
\usepackage{paralist}
\usepackage[hidelinks]{hyperref}
\usepackage{array,multirow}
\usepackage{makecell}
\usepackage{adjustbox}
\renewcommand{\cellalign}{cc}
\usepackage[referable]{threeparttablex}
% \usepackage[toc,page]{appendix}
\usepackage{csquotes}

\usepackage[style=numeric, maxbibnames=10, giveninits=true, uniquename=false, backend=biber, sortcites]{biblatex}
\bibliography{citas}
\usepackage{todonotes}
\usepackage{stmaryrd}
\usepackage{graphicx,xcolor}
\usepackage{subcaption}
\graphicspath{{Imgs/}{Figures/Plots/multiple/}{Figures/Plots/single/}{Figures/}}
\usepackage{amssymb} %simbolos math
\usepackage{amsmath}

%lemmas
\usepackage{amsthm}
\newtheorem{theorem}{Teorema}[section]
\newtheorem{corollary}{Corolario}[theorem]
\newtheorem{lemma}{Lema}[theorem]

%algorithms
\newcommand\mdoubleplus{\ensuremath{\mathbin{+\mkern-10mu+}}}
\newcommand{\zn}[1]{{\protect\color{red}\sf\small #1}}
\newcommand{\za}{\leftarrow}
\newcommand{\zb}{\medskip\noindent$\bullet$\ }
\newcommand{\zdu}{\_\hspace{0.2mm}\_}
\newcommand{\z}[1]{{\scriptsize{\tt #1}:\ \ }}
\newfloat{alg}{th}{tmp}

\newcounter{znoc}
\newcommand{\zno}[1]{\refstepcounter{znoc}\label{#1}\textbf{\ref{#1}}}

\newcommand{\zalg}[1] %TODO poner más espacio entre líneas de algoritmo
{
\begin{figure}[ht]
    % \centering
    % \includegraphics{}
    % \caption{Caption}
    % \label{fig:my_label}

% \vspace{1.5em}
    % \linespread{1.3}
    % \renewcommand{\baselinestretch}{2.0}
\noindent\centerline{
\fbox{
    \normalsize
    \begin{minipage}[l]{8cm}
        \begin{tabbing}
        \textbf{Algorithm} #1
        \end{tabbing}
    \end{minipage}
}}
% \vspace{1.5em}

\end{figure}{}
}

% \setlength{\abovedisplayskip}{1em}
% \setlength{\belowdisplayskip}{1em}

% \newcommand{\zalg}[1]{
% {\begin{alg}
% \footnotesize\noindent\centerline{\fbox{%
% % \footnotesize\bigskip\noindent\centerline{\fbox{%
% \begin{minipage}[l]{8cm}\begin{tabbing}
% \textbf{Algorithm} #1
% \end{tabbing}\end{minipage}}}\end{alg}}}


% Para graficar
\usepackage{tikz}
\usepackage{etoolbox}
\usetikzlibrary{automata, positioning, shapes, arrows, calc}
\tikzset{
    block/.style =  { draw
                    , thick
                    , rectangle
                    , fill=green!10},
    dot/.style  =   { draw
                    , circle
                    , minimum size = .2em
                    , inner sep = 0pt
                    , outer sep = 0pt
                    , fill=black
                },
    }
% \AtBeginEnvironment{tikzpicture}{\shorthandoff{>}\shorthandoff{<}}{}{}

% PP Haskell
\usepackage{listings}
\usepackage{color}
%\lstset{language=Haskell}

% \lstdefinestyle{customhaskell}{
%     breaklines=true,
%     %frame=L,
%     %xleftmargin=\parindent,
%     language=Haskell,
%     showstringspaces=false,
%     %showstringspaces=true,
%     basicstyle=\footnotesize\rmfamily,
%     keywordstyle=\bfseries\color{green!40!black},
%     commentstyle=\itshape\color{blue!40!black},
%     identifierstyle=\color{blue},
%     stringstyle=\color{orange!40!black},
% }
% \lstset{language=Haskell,style=customhaskell}
% \lstset{
%     linewidth=\textwidth,
%     inputencoding=utf8,
%     extendedchars=true,
%     morekeywords={mkVar,mkVars, par, pseq, pars, seqs, TPar,smap},
%     literate=%
%              {→}{{$\rightarrow$}}2
%              {->}{{$\rightarrow$}}2
%              {<-}{{$\leftarrow$}}2
%              {í}{{\'{i}}}1
%              {ó}{{\'{o}}}1
%              {é}{{\'{e}}}1
%              {á}{{\'{a}}}1
%              {ú}{{\'{u}}}1
% }

% \lstnewenvironment{code}[1][]
%     {\minipage{\textwidth}}
%     {\endminipage}

\newcommand{\HRule}{\rule{\linewidth}{0.5mm}}
% \newcommand{\haskell}[1]{\lstinline[breaklines=false]{#1}}
% \newcommand{\phaskell}[1]{\lstinline{#1}}
% \newcommand{\rhaskell}[1]{%
% \lstinline[keywordstyle=\bfseries\color{red},identifierstyle=\color{red},basicstyle=\footnotesize\rmfamily\color{red}]{#1}%
% }

% \usepackage{float}
% \floatstyle{plain}
% \newfloat{program}{t!}{loe}
% \floatname{program}{C\'odigo}

%%%%%%%%%%%%%%%%%%%%%%%% Nombres...
% \newcommand{\nombre}{\textit{Klytius}}
%%%%%%%%%%%%%%%%%%%%%%%% Nombres...

% \makeatletter
% %\AtBeginDocument{%
% \let\c@figure\c@table
% \let\thefigure\thetable
% \let\ftype@table\ftype@figure% give the floats the same precedence
% \let\c@program\c@table
% \let\theprogram\thetable
% \let\ftype@program\ftype@table% give the floats the same precedence
% %\let\c@table\c@program
% %\let\thetable\theprogram
% %\let\ftype@program\ftype@table% give the floats the same precedence
% %\let\c@program\c@figure
% %\let\theprogram\thefigure
% %\let\ftype@figure\ftype@program% give the floats the same precedence
% %}
% \makeatother